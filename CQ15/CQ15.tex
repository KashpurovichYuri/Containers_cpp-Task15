\documentclass[a4paper,12pt]{article}	% тип документа

\usepackage[a4paper,top=1.3cm,bottom=2cm,left=1.5cm,right=1.5cm,marginparwidth=0.75cm]{geometry} % field settings

\usepackage[T2A]{fontenc}		% кодировка
\usepackage[utf8]{inputenc}		% кодировка исходного текста
\usepackage[english,russian]{babel}	% локализация и переносы
\usepackage{indentfirst}

%Piece of code
\usepackage{listings}
\usepackage{xcolor}
\lstset
{
    language=C++,
    backgroundcolor=\color{black!4}, % set backgroundcolor
    basicstyle=\footnotesize,% basic font setting
}

%Drawings
\usepackage{graphicx}

\usepackage{wrapfig}

\usepackage{multirow}

\usepackage{float}

\usepackage{wasysym}

\usepackage[T1]{fontenc}
\usepackage{titlesec}

\setlength{\parindent}{3ex}

%Quatation
\usepackage{csquotes}

% Literature
\addto\captionsrussian{\def\refname{Литература.}}

%Header
\title{
	\center{\textbf{Контрольные вопросы. Задание 15.}}
	}


\begin{document}	% the beginning of the document

\maketitle

\section{В каких ситуациях используются контейнеры типа множество-отображение?}

	Контейнеры типа множество-отображение (ассоциативные контейнеры) могут использоваться в ситуациях, когда необходимо:
	
	\begin{itemize}
	
		\item Довольно быстрый поиск элемента с заданным значением. Дело в том, что обычно ассоциативные контейнеры реализуются в виде бинарных деревьев (или других структур с возможностью автоматической сортировки по ключу), что обеспечивает логарифмическую сложность. В то же время в силу предварительной сортировки	невозможно изменять значения элементов.

		\item Лишь знать, принадлежит ли конкретный элемент рассматриваемой выборке, т.е. находится ли где-то в контейнере. В этом случае следует использовать неупорядоченные ассоциативные контейнеры, в которых элементы не имеют определённого порядка. При этом поиск элемента с конкретным значением выполняется за амортизированную единицу (при "достаточно хорошей" хеш-функции), т.е. ещё быстрее, чем в ассоциативном массиве (т.е. в массиве, имеющем индекс произвольного типа).
		
		\item Чтобы все элементы рассматриваемой коллекции должны быть уникальными, т.е. встречаться единственный раз. В этом случае можно использовать множество или неупорядоченное множество. В свою очередь, соответсвтующие мультимножества используются, когда дубликаты возможны.
		
		\item Построить коллекцию из пар "ключ-значение" с уникальными ключами. Такое может потребоваться для реализации ассоциативного массива, для чего могут быть использованы множетсво или неупорядоченнное множество. В свою очередь, соответствующие мультиотображения могут содержать несколько элементов, имеющих одинаковые значения. Мультиотображения могут использоваться как словари.
	
	\end{itemize}

\section{Каким требованиям должна удовлетворять качественная хеш-функция?}

	Качественная хеш-функция (функция, отображающая значение в хеш-код) должна удовлетворять требованиям:
	
	\begin{itemize}
	
		\item детерминированность, т.е. хеш от одинаковых значений должен оставаться одним и тем же;
		
		\item довольна высокая скорость вычисления, которая не должна зависеть от размера хеш-таблицы, но может зависеть от размера отображаемого значения;
		
		\item равномерность, т.е. значения хеша распределены равномерно;
		
		\item односторонность, т.е. сложность обратного вычисления (по известному хешу);
		
		\item устойчивость к коллизиям.
	
	\end{itemize}
	
	Последние два требования особенно важны для криптографических хеш-функций.

\section{Из-за чего в хеш-таблицах возникают коллизии и как их можно разрешать?}

	Коллизии в хеш-таблицах возникают в связи с тем, что хеш-функции от разных значений может давать одинаковый результат (т.к. хеш-функция отображает множество большей мощности во множетсво меньшей). Коллизии можно разрешать, например, методом цепочек или с помощью открытой адресации.
	
	В методе цепочек при обнаружении коллизии по данному ключу в хеш-таблице добавляется новый элемент, т.е. теперь в таблице будет храниться на одну пару "ключ-значение" с данным ключом больше. При этом на каждое следующее значение ссылается предыдущее. Т.о., формируется последовательный список.
	
	Открытая адресация при обнаружении коллизии, двигать элемент с некоторым шагом по хеш-таблице (размер шага может быть различным в зависимости от реализации), пока не найдётся свободная ячейка, куда и записывается значение. Если все места заняты, то происходит рехеширование. Более того, оно может иметь место и в методе цепочек (чтобы не создавались очень длинные цепочки, затягивающие поиск элементов).

\section{Почему сложность основных операций хеш-таблиц в худшем случае O(N)?}

	Сложность основных операций хеш-таблиц в худшем случае O(N), поскольку неизбежны коллизии, требующие разрешения и при этом в худшем случае:
	
	\begin{itemize}
	
		\item В методе цепочек может образоваться довольно большой список значений (скорее даже, несколько списков) с одинаковым ключом (каждый список -- со своим), а поиск в списках выполняется за O(N) (отсутствует возможность произвольного достпупа к элементам). С этим можно бороться посредством рехеширования, однако оно тоже требует времени не меньше O(N) (хоть и вызывается редко).
		
		\item При открытой адресации произойдёт рехеширование, требующее времени не меньше, чем O(N).
	
	\end{itemize}

\section{В чём заключается преимущество интерфейсов Boost.Multiindex?}

	Boost.Multiindex предоставляет возможность использования различных интерфейсов для одного и того же набора данных. В свою очередь, это может помочь сделать код более быстрым и простым, т.к. появится возможность использовать преимущества различных структур для одной и той же коллекции данных.
	
\newpage


\addcontentsline{toc}{section}{Литература}
 
	\begin{thebibliography}{}
	
		\bibitem{litlink1} Конспект семинара. Макаров И.С.
		\bibitem{litlink2} Standard library. Dzhosattis N.
		\bibitem{litlink3} http://david-grs.github.io/why\_boost\_multi\_index\_container-part1/
		
	\end{thebibliography}


\end{document} % end of the document
